\chapter{Introducere}

\section{Nevoia migrării in cloud}

În era digitală modernă, tendința de procesare mutată în cloud a devenit o parte integrală a multor domenii și industrii. Această tendință se referă la transferul și executarea operațiunilor de procesare a datelor și sarcinilor complexe în cloud, în loc să fie realizate local.

Procesarea mutată în cloud oferă o abordare inovatoare și scalabilă pentru gestionarea volumelor mari de date și sarcini computaționale intensive. Prin intermediul infrastructurii cloud, această tendință permite accesul la resurse computaționale puternice și flexibile, oferind astfel utilizatorilor o experiență îmbunătățită și performanțe superioare în timp real \cite{benefits-of-cloud-computing}.

Prin transferarea sarcinilor în mediul cloud, utilizatorii beneficiază de o serie de avantaje. Unul dintre acestea este accesul la capacități de procesare și resurse de stocare extinse, care pot depăși de multe ori puterea de calcul a dispozitivelor individuale. Această scalabilitate permite utilizatorilor să gestioneze sarcini complexe, cum ar fi analiza unor seturi de date foarte mari, procesarea grafică avansată sau algoritmi de învățare automată, fără a se confrunta cu constrângerile tehnice ale dispozitivelor personale.

Pe lângă avantajele de performanță, procesarea mutată în cloud oferă și o mai mare flexibilitate și mobilitate. Utilizatorii pot accesa și gestiona datele și aplicațiile lor de pe diferite dispozitive, indiferent de locație sau de sistemul de operare utilizat. Acest aspect aduce un nivel crescut de colaborare și sincronizare între utilizatori, sporind eficiența și productivitatea în mediul de lucru modern.

Totuși, pe lângă beneficiile pe care le aduce, procesarea mutată în cloud aduce anumite provocări și preocupări. Una dintre ele este legată de securitatea datelor și confidențialitatea informațiilor. Deoarece datele sunt transferate și procesate în mediul cloud, există riscul ca acestea să fie expuse la amenințări cibernetice sau să fie accesate de persoane neautorizate. Drept urmare, furnizorii de servicii cloud trebuie să implementeze măsuri solide de securitate și criptare pentru a proteja datele utilizatorilor.

\section{Scopul lucrării}

Aplicația este un template care îmbină procese moderne de prelucrare și stocare a datelor folosind servicii cloud. Scopul acesteia este de a fi o implementare practică a unui concept și de a-mi îmbunătăți cunoștiințele cu privire la servicii cloud, arhitectură de aplicații, limbajul C++, clean coding, precum și best practices în acest context. Codul sursă se poate găsi la adresa \href{https://github.com/AndritaLucianGabriel/BachelorsDegree}{https://github.com/AndritaLucianGabriel/BachelorsDegree}.

Alegerea unui limbaj precum C++ pentru dezvoltarea unei aplicații web a prezentat o provocare în ceea ce privește alegerea utilitarului potrivit pentru crearea și gestionarea serverului. Multe aplicații întâlnite folosesc JavaScript și utilitare pentru conectarea la diferiți furnizori de servicii cloud, astfel curiozitatea mea m-a făcut să mă întreb de aplicabilitatea unor alte suite de tehnologii pentru o astfel de aplicație, păstrând totuși ideea centrală de utilizare a cât mai multe servicii cloud pentru diferite functionalități.

\section{Obiective}

In partea de implementare propriu-zisă, am avut în vedere câteva obiective:

\begin{enumerate}
    \item Investigarea caracteristicilor unice ale DialogFlow și motivul pentru care acesta este o alegere potrivită pentru dezvoltarea chatbotilor în sisteme bancare.
    \item Proiectarea și implementarea unui model de chatbot bazat pe DialogFlow, specializat pentru utilizare în sistemul bancar. Acesta ar trebui să fie capabil să răspundă la întrebări frecvente, să ajute clienții să-și gestioneze conturile și tranzacțiile.
    \item Evaluarea performanței modelului de chatbot, în ceea ce privește acuratețea și promptitudinea răspunsurilor, precum și satisfacția generală a utilizatorilor.
    \item Explorarea viitoarelor posibilități de îmbunătățire și adaptare a chatbotilor pentru a răspunde mai bine nevoilor utilizatorilor de servicii bancare.
\end{enumerate}

\section{Motivația personală}

Într-o eră în care tehnologia se dezvoltă cu o rapiditate neegalată, am ales să îmi concentrez lucrarea de licență asupra unui subiect care mă fascinează: arhitectura de tip chatbot în sisteme bancare. Această alegere nu a fost aleatorie, ci a fost alimentată de interesul meu pentru dezvoltarea software, în special în limbajul C++, și de dorința de a explora potențialul aplicațiilor cloud în acest domeniu.

Încă de la începutul studiilor mele, am fost atras de complexitatea și puterea limbajului de programare C++. Acest limbaj mi-a oferit posibilitatea de a construi soluții software robuste, eficiente și flexibile. Dezvoltarea mea personală în acest sens a reprezentat o provocare continuă, dar și o sursă constantă de satisfacție. Alegerea acestei teme pentru lucrarea mea de licență este o oportunitate excelentă de a îmbina abilitățile mele în programarea C++ cu cele în AI și cloud computing.

În ultimii ani, am urmărit cu interes cum organizațiile bancare au început să adopte soluții bazate pe AI pentru a îmbunătăți serviciile pentru clienți și pentru a eficientiza operațiunile interne. Deși există multe implementări de succes (de exemplu ADA \cite{ADA}, asistentul virtual BCR), acest domeniu încă prezintă un potențial incomplet exploatat. Existența acestui potențial mi-a stârnit curiozitatea și m-a determinat să încerc să răspund la următoarea întrebare: este cu adevărat realizabil un sistem de chatbot eficient și complet bazat pe text, fără alt tip de input?

În plus, sunt convins că utilizarea tehnologiei cloud în dezvoltarea acestui chatbot nu va îmbunătăți doar scalabilitatea și disponibilitatea sistemului, dar va permite și implementarea mai ușoară a unor funcționalități avansate de AI. Prin utilizarea serviciului DialogFlow, mă aștept să pot dezvolta un chatbot capabil să îmbunătățească în mod semnificativ interacțiunea dintre bănci și clienții lor.

Așadar, am ales această temă de licență deoarece îmi oferă oportunitatea de a explora aceste idei în detaliu și de a contribui la dezvoltarea soluțiilor tehnologice în domeniul bancar, dar și pentru a-mi deschide și alte porți cu privire la înțelegerea legăturilor dintre servicii cloud independente.

\section{Scurt istoric al integrării asistenților virtuali}

Introducerea asistenților virtuali în sistemele bancare din România a reprezentat un pas semnificativ în modernizarea și digitalizarea serviciilor financiare. Cu o populație tot mai conectată la tehnologie și cu un sector financiar dinamic și inovator, România se aliniază la tendințele globale, îmbrățișând avantajele oferite de inteligența artificială și de tehnologia cloud.

Primii pași în această direcție s-au făcut în ultimii ani, mai exact în 2017, când băncile românești au început să recunoască nevoia de a oferi servicii mai eficiente și mai personalizate clienților lor. Prima bancă românească care a introdus un asistent virtual a fost Banca Transilvania, ea lansând chatbot-ul \emph{Livia} \cite{first-chatbot}. Confruntate cu o concurență acerbă și cu așteptările tot mai mari ale clienților, instituțiile bancare au început să exploreze diferite soluții tehnologice. 

În acest context, asistenții virtuali au apărut ca un instrument promițător pentru îmbunătățirea experienței clientului. Următorul pas pentru Banca Transilvania a fost să migreze în 2019 către platforma Druid, platformă ce se ocupă de dezvoltarea chatbotilor conversaționali. Aceasta tendință a fost urmată și de către BCR, ADA fiind construită tot pe Druid \cite{ADA}.

Asistenții virtuali în serviciile bancare au rolul de a fluidiza interacțiunile cu clienții și de a îmbunătăți calitatea serviciilor. Ei pot răspunde în timp real la o gamă largă de întrebări, pot efectua tranzacții simple în numele clienților și pot oferi asistență personalizată, reducând astfel timpul de așteptare și îmbunătățind satisfacția clientului.

Implementarea asistenților virtuali nu a fost fără provocări. Pe lângă dificultățile tehnice, a fost necesară depășirea reticenței unor clienți de a schimba modul în care interacționau cu orice serviciu bancar, fiind în natura umană de a încerca să nu ieși din zona ta de comfort. În plus, au fost necesare eforturi considerabile pentru a asigura securitatea datelor și pentru a respecta reglementările privind confidențialitatea.

Cu toate acestea, beneficiile pe care le aduc chatbotii în sistemul bancar sunt semnificative. În primul rând, ei pot funcționa non-stop, asigurând asistență clienților în orice moment al zilei sau al nopții. În al doilea rând, pot gestiona un volum mare de solicitări simultan, ceea ce este dificil de realizat pentru angajații umani. În al treilea rând, folosirea chatbotilor poate reduce costurile operaționale, întrucât necesită mai puține resurse umane.

Deși este încă la început, implementarea asistenților virtuali în sistemele bancare din România arată promițător. Cu un nivel din ce în ce mai mare de acceptare și cu progresele continue în tehnologia AI, chatbotii vor ajunge să joace un rol tot mai important în transformarea digitală a sectorului bancar românesc și nu numai.

\section{Structura lucrării}

Lucrarea este structurată în următoarele capitole:

\begin{enumerate}
    \item \textbf{Introducere}: prezintă o viziune generală asupra conținutului lucrării, punând accent pe tendința modernă de mutare a procesării în cloud cât și automatizarea proceselor prin chatboti conversaționali.
    \item \textbf{Preliminarii}: explică elementele cheie ale lucrării și subdomeniului de care aceasta aparține.
    \item \textbf{Ecosistemul Google Cloud}: descrie modul de funcționare al platformei și beneficiile utilizării sale.
    \item \textbf{Implementare}: această secțiune.
    \item \textbf{Explicațiile codului din cloud}: această secțiune.
    \item \textbf{Explicațiile codului local}: această secțiune.
    \item \textbf{Concluzii}: această secțiune sintetizează informațiile discutate și propune posibile direcții de dezvoltare pentru viitor.
\end{enumerate}