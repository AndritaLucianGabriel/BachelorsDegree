\chapter{Ecosistemul Google Cloud}

\section{Alegerea platformei agentului}

Comparând Dialogflow cu alte servicii de agenți conversaționali, există mai multe caracteristici care o diferențiază și recomandă pentru utilizarea în sectorul bancar.

Dialogflow, deținut de Google, este o platformă avansată pentru dezvoltarea de aplicații de conversație, care folosește tehnologia AI pentru a interpreta intențiile și contextul utilizatorului \cite{google_dialogflow}. Aceasta oferă o gamă largă de funcționalități, inclusiv integrarea cu diverse platforme de mesagerie, asistenți virtuali și alte servicii Google, cum ar fi Google Cloud Functions.

La rândul lor, serviciile alternative, cum ar fi IBM Watson, Amazon Lex și Microsoft Luis, prezintă și ele avantaje. IBM Watson se remarcă prin puterea sa de a învăța în mod continuu și de a se adapta la diverse contexte de utilizare \cite{ibm_watson}. Amazon Lex beneficiază de integrarea nativă cu ecosistemul Amazon Web Services (AWS), oferind posibilități extinse de dezvoltare și scalare \cite{amazon_lex}. Între timp, Microsoft Luis are avantajul integrării strânse cu suita de produse Microsoft, incluzând Office 365 și Azure \cite{microsoft_luis}.

Cu toate acestea, Dialogflow se distinge prin mai multe aspecte-cheie. În primul rând, Dialogflow este foarte flexibil, permițând dezvoltatorilor să creeze experiențe de conversație personalizate pentru diferite platforme și canale de comunicare. Acesta poate fi integrat cu o multitudine de servicii, de la Google Assistant și Amazon Alexa, până la Facebook Messenger și Slack.

În al doilea rând, Dialogflow este folosește servicii integrate în ecosistemul Google Cloud. Acesta permite dezvoltatorilor să creeze, să testeze și să implementeze chatboti direct în cloud, profitând de avantajele cloud computing, inclusiv scalabilitatea, redundanța și accesul la cele mai recente inovații AI.

Aici intervine Google Cloud Functions \cite{google_cloud_functions}, un serviciu de calcul care permite dezvoltatorilor să execute cod ca răspuns la evenimente specifice, fără a fi nevoie să administreze o infrastructură de server. Acest serviciu poate fi utilizat în tandem cu Dialogflow pentru a crea funcții de backend pentru chatbot, cum ar fi procesarea cererilor utilizatorului, integrarea cu alte sisteme sau baze de date, sau gestionarea autentificării și a securității.

Google Cloud Functions se integrează perfect cu Google Source Repositories \cite{google_source_repositories}, un serviciu de găzduire de cod sursă care oferă un loc sigur și scalabil pentru a stoca și a gestiona codul. Acest lucru permite dezvoltatorilor să colaboreze eficient la proiecte, să gestioneze versiunile de cod și să implementeze automat codul în Cloud Functions.

În final, Google Cloud Storage oferă un serviciu de stocare de obiecte scalabil și durabil, care poate fi utilizat pentru a stoca și a servi datele utilizate de chatbot, cum ar fi înregistrări de conversații, profile de utilizator, sau alte date de context \cite{google_cloud_storage}.

\begin{figure}[h]
    \centering
    \includegraphics[width=1.0\textwidth]{fulfillment-flow}
    \caption{Flow-ul intern al Dialogflow \cite{google_dialogflow}}
\end{figure}

În ansamblu, alegerea Dialogflow, împreună cu Google Cloud Functions, Google Source Repositories și Google Cloud Storage, oferă o soluție robustă și flexibilă pentru dezvoltarea de chatboti în sectorul bancar. Prin folosirea acestor tehnologii, băncile pot crea experiențe de conversație personalizate, eficiente și securizate pentru clienții lor.

\section{Cum funcționează ecosistemul}

Dialogflow folosește input-ul trimis de către Dialogflow API C++ Client \cite{dialogflow_client_library}, acesta fiind parsat și este trecut prin verificarea lor internă cu intențiile create în prealabil\footnote{O intenție reprezintă un anumit rezultat pe care doriți să îl obțineți de la interacțiunea utilizatorului. De exemplu, o intenție poate fi „programare întâlnire” sau „informații despre cont”.}. În funcție de cum este creat intent-ul și scopul său, pot exista parametrii scoși sub formă de entități\footnote{Entitățile sunt concepte valoroase care pot fi extrase din declarațiile utilizatorilor. De exemplu, în declarația „Doresc să programez o întâlnire pentru marți”, „marți” este o entitate de tip „dată”.} din textul primit (sau este un intent default, cu rol de legătură între altele cu functionalități).

\begin{figure}[h]
    \centering
    \includegraphics[width=1.0\textwidth]{entitati}
    \caption{Reprezentarea unei entități pentru IBAN folosind un regex}
    \label{fig:entitati}
\end{figure}

În DialogFlow, o "intenție" reprezintă un anumit rezultat pe care îl doriți de la o interacțiune cu utilizatorul. Atunci când un utilizator trimite un input (cum ar fi o întrebare sau o comandă), DialogFlow potrivește inputul cu cea mai bună intenție pe baza a ceea ce ați setat în agentul dvs. În cadrul unei intenții, există mai multe câmpuri și concepte cheie care sunt utilizate pentru a defini și a rafina comportamentul intenției.

\textbf{Contextele} permit agentului DialogFlow să înțeleagă starea conversației și să răspundă în mod corespunzător. Există două tipuri de contexte: contexte de intrare și contexte de ieșire. Contextele de intrare sunt cele pe care agentul le caută înainte de a potrivi o intenție, în timp ce contextele de ieșire sunt stabilite după ce o intenție este potrivită.

\textbf{Exemple de declarații ale utilizatorului} sunt exemple de ceea ce utilizatorul ar putea spune pentru a declanșa această intenție (vezi Figura \ref{fig:training-phrases}). Sistemul utilizează aceste exemple pentru a învăța modelul de limbaj pentru a recunoaște aceeași intenție din declarații diferite.

\begin{figure}[H]
    \centering
    \includegraphics[width=1.0\textwidth]{training-phrases}
    \caption{Frazele de antrenament ale botului}
    \label{fig:training-phrases}
\end{figure}

\textbf{Răspunsurile} reprezintă modul de interacționare hard-codată a agentului cu inputul clientului. Astfel, un răspuns toate fi considerat fie final de conversație, fie poate avea un răspuns default, folosind functionalitatea webhook-ului ulterior.

\begin{figure}[H]
    \centering
    \includegraphics[width=1.0\textwidth]{fulfillment}
    \caption{Selectarea fulfillment-ului}
    \label{fig:fulfillment}
\end{figure}

În cadrul platformei există niște entități predefinite \cite{system-entities} pentru a ușura munca utilizatorului atunci când vine vorba de extragere de parametrii, dar functionalitatea puternică al acestei opțiuni este extragerea parametrizată a informațiilor. Dupa cum putea vedea in Figura \ref{fig:actiune-parametrii}, acțiunea propriu-zisă oferă flexibilitate atât la prezența parametrului (daca acesta se poate scoate din text), cât și la un text default dacă acesta nu a fost găsit. Coloana \emph{value} reprezintă numele variabilei care este trimisă către metoda din Google Cloud Function, prin webhook-ul definit în tabul de \emph{fulfillment}.

\begin{figure}[h]
    \centering
    \includegraphics[width=1.0\textwidth]{actiune-parametrii}
    \caption{Extragerea de paramertii din inputul clientului}
    \label{fig:actiune-parametrii}
\end{figure}

În tab-ul de fulfillment putem găsi modul de procesare al informației extrase de către agent. În cazul de față, am folosit un webhook extern, care folosește integrarea cu Google Cloud Function. Astfel, parametrii scoși vor fi trimiși sub forma unui request către acel URL. Detaliile acelui request se află în anexa \ref{annex:request}.

Tabul de "Validation" din DialogFlow ES are de-a face cu revizuirea și aprobarea sau respingerea propunerilor pe care DialogFlow le face pentru îmbunătățirea modelului asistentului virtual. Acesta este un element al învățării automate interactive, unde sistemul învață din feedbackul dat de utilizator.

După ce asistentul virtual a fost folosit o vreme, acesta va începe să învețe din interacțiunile cu utilizatorii și va încerca să îmbunătățească precizia detecției intențiilor și a entităților. Acest lucru este realizat prin generarea de "sugestii" pe baza interacțiunilor anterioare. Aceste sugestii apar în tabul "Validation".

Fiecare sugestie conține următoarele elemente:

Training phrase - Acesta este textul interacțiunii dintre utilizator și asistentul virtual.

Intent - Acesta este numele intenției pe care DialogFlow o propune pentru fraza de instruire. Acesta poate fi o intenție existentă sau o nouă intenție sugerată de DialogFlow.

Action - Acesta este un câmp opțional care permite asocierea unei acțiuni cu o intenție. DialogFlow poate sugera o acțiune pe baza contextului frazei de instruire.

Entities - Acestea sunt entitățile pe care DialogFlow le propune pentru a fi asociate cu fraza de instruire.

Pentru fiecare sugestie, puteți alege să o aprobați sau să o respingeți (vezi Figura \ref{fig:validation}). Dacă alegeți să aprobați sugestia, DialogFlow va actualiza modelul asistentului virtual pentru a include informațiile din sugestie, îmbunătățind astfel precizia detecției de intenții și entități în interacțiunile viitoare. Dacă respingeți sugestia, DialogFlow nu va face nicio modificare.

\begin{figure}[h]
    \centering
    \includegraphics[width=1.0\textwidth]{validation}
    \caption{Validarea rezultatului agentului}
    \label{fig:validation}
\end{figure}