\chapter{Setup și compilare}

\section{Instalarea bibliotecilor}
Utilizatorul trebuie să aibă o cheie de acces pentru un cont de serviciu pe platforma Google Cloud pentru a accesa API-ul Dialogflow ES. Această cheie trebuie plasată sub directorul \emph{res}. Mai multe informații găsiți \href{https://cloud.google.com/iam/docs/keys-create-delete}{aici}.

Această cheie va fi setată ca variabilă de mediu \texttt{GOOGLE\_APPLICATION\_CREDENTIALS}, care va funcționa ca un sistem de autentificare pentru platformă.

În directorul rădăcină al proiectului, se află un script Bash numit \texttt{getting\_started.sh} (Notă: dacă scriptul nu poate fi rulat pe o platformă Unix, recomandăm verificarea formatului greșit al sfârșitului de linie prin executarea comenzii \texttt{dos2unix} pentru script și verificarea din nou). Când este rulat pe o platformă Unix, acesta va instala toate dependențele necesare (vezi \hyperref[first-time-setup]{Configurarea inițială}). După finalizarea scriptului, utilizatorul trebuie să ruleze manual configurarea inițială pentru CLI-ul GC:

\begin{verbatim}
./google-cloud-sdk/install.sh
source ~/.bashrc
./google-cloud-sdk/bin/gcloud init
\end{verbatim}

Notă: Configurarea inițială va dura un timp \textbf{îndelungat} (în special pe mașini mai vechi).

\section{Configurarea inițială}
Fluxul scriptului \texttt{getting\_started.sh} este următorul:

\textbf{Notă importantă}: Trebuie să fie rulat cu opțiunea \texttt{-i} sau \texttt{--install}, iar după finalizarea acestuia, utilizatorul trebuie să ruleze manual configurarea inițială pentru gcloud (vezi pașii de mai sus).

\begin{enumerate}
  \item Configurarea mediului:
    \begin{enumerate}
      \item Submodulele sunt inițializate.
      \item Pe baza diferenței în terminarea greșită a liniei, atât scriptul \texttt{build.sh}, cât și executabilul \texttt{sass} vor fi convertite la formatul Unix-like.
      \item Se instalează \texttt{doxygen} și dependența sa \texttt{graphviz}.
    \end{enumerate}
  \item Revenirea la versiunile dorite ale bibliotecilor:
    \begin{enumerate}
      \item Se utilizează versiunea \texttt{Poco v1.12.4}.
      \item Se utilizează versiunea \texttt{VCPKG v2023.04.15 Release}.
      \item Se utilizează versiunea \texttt{GCP v2.9.1}.
    \end{enumerate}
  \item Configurarea Poco. Mai multe informații găsiți \href{https://github.com/pocoproject/poco}{aici}:
    \begin{enumerate}
      \item Se instalează dependențele (\texttt{openssl, libssl-dev, git, g++, make, cmake}).
      \item Se configurează fișierul \texttt{CMakeLists.txt}.
      \item Se construiește fișierul de configurare.
      \item Se instalează biblioteca.
    \end{enumerate}
  \item Configurarea npm:
    \begin{enumerate}
      \item Se instalează \texttt{npm} prin intermediul gestionarului de pachete implicit.
    \end{enumerate}
  \item Configurarea Google Cloud CLI. Mai multe informații găsiți \href{https://cloud.google.com/sdk/docs/install}{aici}:
    \begin{enumerate}
      \item Se descarcă arhiva care conține codul sursă.
      \item Se extrage arhiva.
    \end{enumerate}
  \item Configurarea Sass:
    \begin{enumerate}
      \item Se creează un link simbolic (\texttt{symlink}) și se forțează crearea acestuia, chiar dacă există, către executabilul furnizat (vezi directorul \texttt{libs/sass}).
    \end{enumerate}
  \item Configurarea VCPKG. Mai multe informații găsiți \href{https://github.com/Microsoft/vcpkg}{aici}:
    \begin{enumerate}
      \item Se instalează dependențele (\texttt{curl, zip, unzip, tar}).
      \item Se configurează managerul vcpkg pentru prima dată prin intermediul scriptului de pornire (\texttt{libs/vcpkg/bootstrap-vcpkg.sh}).
    \end{enumerate}
  \item Configurarea Google Cloud Platform. Mai multe informații găsiți \href{https://github.com/googleapis/google-cloud-cpp/tree/main/google/cloud/dialogflow_es/quickstart}{aici}:
    \begin{enumerate}
      \item Se instalează modulele reale pe care le vom utiliza în proiect (\texttt{core} și \texttt{dialogflow-es}) prin intermediul vcpkg.
    \end{enumerate}
\end{enumerate}

\section{Cum se construiește}
Din directorul rădăcină al proiectului, putem utiliza scriptul \texttt{build.sh} pentru a construi și rula aplicația. În mod implicit, serverul va utiliza portul \textbf{9090}, dar acesta poate fi modificat din fișierul \texttt{constants.h} (vezi \texttt{globals::serverPort}).

Există câteva opțiuni pe care le puteți utiliza cu scriptul:
\begin{itemize}
  \item \texttt{-b} sau \texttt{--build} => compilează și rulează proiectul
  \item \texttt{-bscss} sau \texttt{--build-scss} => compilează doar fișierele SCSS pentru a actualiza aspectul HTML-ului
  \item \texttt{-docs} sau \texttt{--generate-docs} => generează documentația tehnică utilizând Doxygen
  \item \texttt{-c} sau \texttt{--clean} => curăță proiectul
  \item \texttt{-v} sau \texttt{--verbose} => opțional, afișează mai multe informații în consolă
  \item \texttt{-h} sau \texttt{--help} => afișează această informație de ajutor
\end{itemize}

Notă: Nu rulați opțiunea \texttt{-c} împreună cu alte opțiuni, deoarece curățarea trebuie să fie un proces singular.